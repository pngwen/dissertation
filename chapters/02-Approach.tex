\documentclass[../dissertation.tex]{subfiles}
\begin{document}

\chapter{Approach}
This chapter presents the tensor document model used to discover
influencing factors.  In the first section, enough background
information is given to familiarize the reader with tensors and tensor
decompositions. Following this section, the specific formulation of
the document tensors are presented.  The final two sections of this
chapter cover the complete influence model as well as a justification
for the model and its interpretations.

\section{Tensors and Decompositions}
The term ``tensor'' can carry several interpretations.
Mathematically, a tensor is a geometric object which represents
a multilinear system.  Tensors are an abstraction of other linear
maps, extending them into arbitrary numbers of modes.  The number of
modes in a tensor correspond to the number of indices needed to
reference the components within the tensor.  As such, the familiar
bottom-up hierarchy of {\em scalar, vector, matrix}, is really 
a hierarchy of tensors.  A {\em scalar} has zero modes, a {\em vector}
has one mode, and a {\em matrix} has two modes.  Tensors can be
represented as the outer product of vector spaces, the outer product
of other tensors, or as multidimensional arrays. Table
\ref{tab:tensor} shows various types of tensors, and also demonstrates
the conventions which will be used in the remainder of this
dissertation.  Note that in Table \ref{tab:tensor}, the tensors are
assumed to have an applied reference basis, which is what allows them
to be written as arrays.  


\begin{table}
    \centering
    \begin{tabular}{c|c|c|c}
        {\bf Modes} & {\bf Name} & {\bf As Tensor Product} & {\bf Array}\\
        \hline
        0 & Scalar & n/a & n/a\\
        1 & Vector & $\vec{v}$ & $\vec{v}_i$\\
        2 & Matrix & $M=\vec{v} \otimes \vec{w}$ & $M_{ij}$\\
        3 & Tensor & $T=\vec{v} \otimes \vec{w} \otimes \vec{y}$ & $T_{ijk}$ \\
        $n$ & Tensor & $T=\vec{v_1} \otimes \vec{v_2} \otimes \ldots \otimes \vec{v_n}$ & $T_{i_1i_2 \ldots i_n}$\\
    \end{tabular}

    \caption{Tensors of Various Orders}
    \label{tab:tensor}
\end{table}

For the purposes of stasticial analysis, tensors are generally thought
of as multi-dimensional arrays. Of particular interest is the polyadic
form of tensors, first described in~\cite{hitchcock1927}. 
\section{Tensor Construction}
\subsection{Corpus Vocabulary}
\subsection{Frequency Counting}

\section{Decomposition and Normalization}
\subsection{PARAFAC}
\subsection{Factor Weights}

\section{Influence Modelling}
\subsection{Factor Comparisons}
\subsection{Factor Weighting}
\subsection{Influence Vector}

\section{Justification of the Model}
\subsection{Measurements and Structural Expectations}
\subsection{Proportional Profiles and Implied Axes}
\subsection{Measuring Fit}
\end{document}
