\documentclass[../ut-dissertation.tex]{subfiles}
\begin{document}

\chapter{Conclusions}
This chapter contains a summary of the results, a justfication of the
model, an outline of various weaknesses of the model, and then
concludes with a discussion of future work.

\section{Model Performance}
As can be seen from the experiments in the previous chapter, the model
performs as expected.  The factors discovered by non-negative tensor
decomposition all contain related $n$-grams.  Moreover, the factors
are unique enough that a match, or a near match within some heuristic
bound, provides evidence of a relationship between factors.  The
related factors also make sense on an intuitive level, each having a
clear semantic relationship to both target and source material.

The quantification of the influencing factors also perform as
expected.  While no ground truth is available for a weighted mixture
of source documents on the target documents, inspecting the documents
in the corpus shows that the model's output matches the expectations
of a human reader.  


\section{Justification of the Model}
This influence model is based on a factorization of tensors describing
the $n$-gram frequency counts of the document.  $n$-grams are a fairly
common approach to modeling the topic and style of documents, and
therefore the topic and styles of the documents can logically be said
to be contained within the tensor representation of them.  By
decomposing a document tensor into a non-negative polyadic
decomposition, the resultant factors will be a mixture of covariant
and contravariant factors.  That is, the frequencies of the $n$-grams
found within each factor will have similar covariant properties.  This
is done irrespective of the order of the $n$-grams within the
document.

The next step of the model normalizes all the factors.  In so doing,
this removes the magnitude of the frequencies and leaves the factor
tensors with a proportional profile of the $n$-gram composition of the
document.  This factor now contains a description of each of the
principal elements of the document in relationship to its vocabulary.

By repeating this process for every document in the corpus, this
technique produces a set of proportional profiles that describe the
make up of each document within the corpus.  By searching for
comonality among these factors, the influence model locates documents
which have common explanatory factors.  If a document has a strong
enough match on one or more of its factors, it provides evidence of a
relationship between the documents.  This evidence can be considered
to vary in strength according to the selection heuristic, and because
the factors are based on the distribution of phrases within the
document, this provides a sound model of document influence.

Another aspect of this operation that makes this a useful model is
that it compares all factors irrespective of their influence in their
original source document.  That is to say if a source puts forward
some topic, which is subsequently modeled as a proportional factor,
and that topic is relatively unimportant in its source this will have
no impact on how much influence it may exert in a target document.  In
traditional influence models, which are based on word frequency, this
sort of relationship will not be found.  However, with the present
model it will, and its weight will be based upon the total impact it
has on the target document. 

\section{Weaknesses of the Model}
Perhaps the greatest weakness of the present model is that no ground
truth is available.  This sort of quantification of influence is not
readily available, but given the justification of the model the
information it discovers can still be considered useful.

Another problem with this model is that it is sensitive to three input
parameters.  $n$ for the number of words, $nfactors$ for the number of
decomposition factors and $threshold$ for the selection criteria.  Of
course, justifications for the selection of $n$ and $threshold$ have
already been discussed.  In the case of $n$, the setting of 3 is a
standard starting point in the field of $n$-gram analysis for English,
but this would likely need to be tuned for other languages.  The
biggest impediment to succesfully modelling influence in this fashion
is the $nfactors$.  In order to get a well-fit and unique document
tensor decomposition, the rank of the tensors is needed.  However,
knowing the true value of this parameter is intractible and it must be
searched for.  Further study and research into the open question of
the rank of sparse tensors would alleviate this problem.


\section{Future Research}
The immediate future plans for this research involve the further
development of the sptensor library.  Further optimization is needed,
as well as completion of its MPI interface.  Following that, library
bindings to higher level languages will be created.

Another application of the model is in the works which will replicate
the studies conducted by Craig, Hugh, and Burrows~\cite{burrows2017,
  craig2009}.  Addressing the problem of Shakespearean authorship
using this model poses several unique challenges, not least of which
is due to the inconsistencies in Elizabethan spelling (which
necessitates the decomposition over full vocabularies).  Additional
applications will also be explored, including establishing
chronologies of documents via topological sorting and modeling the
influence flow through a hierarchical network of documents.

The effects of constrained vocabularies are another area which needs
to be addressed.  Following the authorship studies, another future
effort will be to address the effect of the vocabulary size on the
output of the model.  Other aspects warranting further study are the
effects of the various tunable parameters of the model.  
\end{document}
