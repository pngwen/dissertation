\documentclass[../dissertation.tex]{subfiles}
\begin{document}

\chapter{Introduction}
\begin{quote}
\textit{Nam cum pictor praecogitat quae facturus est, habet quidem in
intellectu sed nondum intelligit esse quod nondum fecit.}

-- Anselm of Canterbury~\cite{anselm}
\end{quote}
In the eleventh century, Anselm of Canterbury wrote what has since
come to be known as the ontological argument for the existence of
God~\cite{anselm}.  Anselm's argument was based on the assumption that
all ideas, or more specifically, all thoughts originate either from
perceptions of the outside world or from images formed within the
imagination.  From this he provides an argument for the existence of
a divine being.  The research presnted here follows this same
epistemologcal assumption to a much less trivial end.
Instead of proving divine influence, the present work shall attempt to
measure the influence present in written works of less divine beings.

The need for such a measurement should be readily apparent to anyone
working in any academic field.  In modern academia, the value of any
given participant is rooted in an attempt to measure that person's
influence over their own field.  Traditional approaches to this
problem involve simply counting citations~\cite{**Use and Abuse**},
while more modern approaches tend to involve some document
semantics~\cite{**some semantic influence models**}.  In addition to
academic evaluation, the present work should also prove useful to the
study of any sort of text.  This will have applications in establish
authorship and chronology of text documents.  Further possible
extensions of these techniques will be given in the final chapter of
this dissertation.

\section{Modelling Influence}
The influence model presented in this dissertation makes the
assumption that the text of a written document is the result of
a process which takes two primary inputs.  The first is the documents
which an author has read, and the second is the contributions of the
author's own thought processes.  The influence exerted on a document
which results from this process is assumed to be present in the form
of topics, word choice, and phrase structure.  Stated another way, the
influencing factors are to be interpreted as the degree to which one
document ``sounds like'' another.  Intuitively, a human reader is able 
to produce a qualitative model from reading source material and
a target document.  

** TODO: Get some info from the shakespeare authorship companion for
shakespeare and Marlowe **

The basic steps for constructing this model are as follows:
\begin{enumerate}
    \item Find explanatory factors for all documents (sources and
        target).
    \item Find common explantory factors between source and target
        documents.  
    \item Any factor of the target document not matched to factors
        form source documents are attributed to the author.
    \item Assign weights to all explantory factors.
\end{enumerate}



\section{Related Work}

\end{document}
