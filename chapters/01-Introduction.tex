\documentclass[../dissertation.tex]{subfiles}
\begin{document}

\chapter{Introduction}
\begin{quote}
\textit{Nam cum pictor praecogitat quae facturus est, habet quidem in intellectu sed nondum intelligit esse quod nondum fecit.}

-- Anselm of Canterbury~\cite{anselm}
\end{quote}
In the eleventh century, Anselm of Canterbury wrote what has since
come to be known as the ontological argument for the existence of
God~\cite{anselm}.  Anselm's argument was based on the assumption that
all ideas or, more specifically, all thoughts originate either from
perceptions of the outside world or from images formed within the
imagination.  From this he provides an argument for the existence of
a divine being.  The research presnted here follows this same thread
of logic to a much less trivial end.  Instead of proving divine
influence, the present work shall attempt to measure the influence
present in the written works of man.  

The need for such a measurement should be readily apparent to anyone
working in any academic field.  In modern academia, the value of any
given participant is rooted in an attempt to measure that person's
influence over their own field.  Traditional approaches to this
problem involve simply counting citations~\cite{**Use and Abuse**},
while more modern approaches tend to involve some document
semantics~\cite{**some semantic influence models**}.  In addition to
academic evaluation, the present work should also prove useful to the
study of any sort of text.  This will have applications in establish
authorship and chronology of text documents.  Further possible
extensions of these techniques will be given in the final chapter of
this dissertation.

\section{Modelling Influence}

\section{Related Work}

\section{Tensors and Decompositions}
\end{document}
