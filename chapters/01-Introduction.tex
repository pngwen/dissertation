\documentclass[../dissertation.tex]{subfiles}
\begin{document}

\chapter{Introduction}
\begin{quote}
\textit{Nam cum pictor praecogitat quae facturus est, habet quidem in
intellectu sed nondum intelligit esse quod nondum fecit.}

-- Anselm of Canterbury~\cite{anselm}
\end{quote}
In the eleventh century, Anselm of Canterbury wrote what has since
come to be known as the ontological argument for the existence of
God~\cite{anselm}.  Anselm's argument was based on the assumption that
all ideas, or more specifically, all thoughts originate either from
perceptions of the outside world or from images formed within the
imagination.  From this he provides an argument for the existence of
a divine being.  The research presnted here follows this same
epistemologcal assumption to a much less trivial end.
Instead of proving divine influence, the present work shall attempt to
measure the influence present in the written works of less divine beings.

The basic assumption made about text documents is the same assumption
that Anselm made about the origin of thoughts.  Every word, phrase,
sentence, paragraph, and theme in a document must come from one of two
sources.  Either the author created the thought from within their own
mind, and as such this counts as a literary contribution, or the
author could have transferred ideas from some outside source.  These
sources can take on many forms.  In the case of academic writing, the
author is likely to have been influenced primarily by the various
books and papers that they have read over the course of their
research.  Of course, another form of influence is a coauthor (though
in the case of academic literature, coauthors are almost always
explicitly stated.)  Of course, the influence over the text in a paper
is not constrained merely to the literature that the author has cited,
but is ultimately a reflection of an author's entire life experience
and background.  In the case of literary writing, such as a novel or
play, a reasonable assumption is that an author is influenced by other
works within their genre as well as by the society in which they live.

Given that every written document is influenced by at least a small
set of outside documents, the present work attempts to model and
quantify this influence by separating documents into a set of factors
and then searching for common factors among the documents.  The
desired result has two parts.  First, a weight is assigned to each
factor indicating its importance in the target work.  Second, the
factors themselves should carry enough semantic meaning to identify
the ideas and elements of style which have been transferred from a
source document to a target document.  Thus, the goal of the present
work is to identify influencing factors and to quantify the influence
they exert on a target document.

The usefulness of such a measurement should be readily apparent to
anyone working in any academic field.  In modern research, the
perofrmance of participants is rooted in an attempt to measure that
person's influence over their chosen field.  Traditional approaches to
this problem involve counting citations over a specific window of time
~\cite{adler2009} while more modern approaches tend to involve some
document semantics~\cite{dietz2007, jiang2014, **buneman**} Measuring
influence in a written document can also be applied in situations
where authorship is in question.  Given a corpus of works of confirmed
provenance, and a disputed document, influence modelling can identify
the possible influence of each author.  Thus textual influence
modelling can be used to answer the question of authorship where it is
disputed, or could potentially be used to identify plagearised
passages.


\section{Modelling Influence}
At a high level, the goal is to model the extent to which one document
``sounds like'' another and to identify elements which would lead a
reader to that conclusion.  Human readers seem to be able to perform
this operation on an intuitive level, though they are typically not
aware of the elements which lead them to their
conclusions~\cite{craig2009}.  Moreover, a human reader cannot
typically quantify the relationships they perceive among several
authors as their input is more qualitative.  Take for instance the
following quotes:

% Marlowe Quote
% Shakespeare Quote
% Trump Quote ``I have the best words''
% http://www.kansascity.com/news/local/news-columns-blogs/the-buzz/article55604115.html

Clearly, the first two have similar elements of style and word use.
Both quotes indicate a high mastery of language as well as a certainly
outmoded style of speech.  The final quote, however, exhibits none of
these qualities.  It is decidedly modern and is also lacking in style or
wit.  


** TODO: Get some info from the shakespeare authorship companion for
shakespeare and Marlowe **

The basic steps for constructing this model are as follows:
\begin{enumerate}
    \item Find explanatory factors for all documents (sources and
        target).
    \item Find common explantory factors between source and target
        documents.  
    \item Any factor of the target document not matched to factors
        form source documents are attributed to the author.
    \item Assign weights to all explantory factors.
\end{enumerate}



\section{Related Work}

\section{Bibliography (temporarily in each chapter)}
\bibliographystyle{unsrt}
\bibliography{../sources}
\end{document}
