%% LyX 2.0.2 created this file.  For more info, see http://www.lyx.org/.
%% Do not edit unless you really know what you are doing.
\documentclass[english]{article}
\usepackage[T1]{fontenc}
\usepackage[latin9]{inputenc}
\usepackage{babel}
\begin{document}

\section{Introduction}

The purpose of these proposed techniques is to explore extend the
remaining useful life (RUL) of a system through the use of prognostics
and adaptive control. The overall approach consists of two parts.
The first is an RUL predictor, and the second is control modification.
The prognostic predictor is used to extract information which is used
to augment the system's control model. This will allow the system
to modify its behavior as it progresses through degraded states of
operation in order to maximize its remaining useful life.

The remainder of this document describes a proposed prognostics algorithm,
a framework to be used to augment control, and a proposed set of experiments
for validating both the prognostics algorithm and the control modification
steps.


\section{Prognostics Approach}

Many approaches exist for performing prognostic prediction. A generalized
framework for these many methods will allow system life extension
to be explored in as broad and general method as available. Also,
by unifying the process of performing predictions, multiple predictors
can be used in conjunction with other system elements (such as the
adaptive control layer). This generalized approach consists of a vector
space representation of the system, a set of functions for mapping
states onto the vector space, and a method for performing the predictions.


\subsection{Vector Space Formulation}

The vector space is formulated as follows:
\begin{description}
\item [{$\mathcal{F}$}] A complex-valued $k$ dimensional vector space
where $k$ is some constant.
\item [{$F$}] A function which maps sensor history $S$ onto some vector
$\bar{v_{s}}\in\mathcal{F}$
\item [{$R$}] A set of regions within $\mathcal{F}$ which indicate system
failure.
\item [{$D$}] A function which maps $\bar{v}_{s}$onto a real value $d$
which is the distance from the nearest failure region $r\in R$  
\end{description}

\subsection{Performing Predictions}


\subsection{Training Methods}


\subsection{Proposed Advantages}


\section{Corrective Action}


\section{Experiments}
\end{document}
