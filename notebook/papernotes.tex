\documentclass{article}
\title{Paper Notes}
\author{Robert Lowe}

\begin{document}
\maketitle

\section{Unsorted Papers}

\subsection{Complex-networks-Structure and dynamics}
\begin{itemize}
  \item {\bf Title:} Complex networks: Structure and dynamics
  \item {\bf Authors:} S. Boccaletti, V. Latorab, Y. Morenod, M. Chavezf, U. Hwanga
  \item Physics Reports 2006
  \item Study of how networks of dynamix systems interact.
  \item Graph Theory based.  Largely an overview of weighted directed graphs.
  \item Opinion Formation
  \begin{itemize}
    \item The most interesting application for me in this one is opinion formation.
    \item Ising Models Majorona
    \item Opnion influence models: Sznajd, Deffuant, Krause and Hegselman
  \end{itemize}
  \item Papers to Read
  \begin{itemize}
    \item E Majorana, Scientia 36 (1942) 58
    \item K Sznajd-Weron Mod. Physics C11 (2000) 1157
    \item R. Heggelmann, U. Krausse Arti Soc. Social Simul 5(3) (2002) paper
    2 jass.soc.surrey.uk    
  \end{itemize}
\end{itemize}

\subsection{Growth and structure of slovenia }
\begin{itemize}
  \item{\bf Title:} Growth and structure of Slovenia’s scientific collaboration network
  \item{\bf Author:} Matjaz Perc
  \item Interesting paper on small world problem, largely based on graphs of known associations
  \item Studies how collaboration networks evolve over time
  \item May be a useful topic to pursue once I can establish what the associative networks are.
\end{itemize}

\subsection{Scale free characteristics of random networks}
\begin{itemize}
  \item {\bf Title:} Scale-free characteristics of random networks:
the topology of the world-wide web
  \item {\bf Authors: } Albert-Laszlo Barabasi, Reka Albert, Hawoong Jeong
  \item Extracts self-organizing order in complex networks, esp. www
  \item Power Law connectivity
  \[
  P(k)\sim k^{-\gamma}
  \]
  \item Properties of distribution of large networks
  \item Interesting, but not directly relavent to me!
\end{itemize}


\subsection{inducible regularization for low rank matrix}
\begin{itemize}
  \item {\bf Title: } Inducible regularization for low-rank matrix factorizations for collaborative filtering
  \item {\bf Authors: } Zhenyue Zhang, Keke Zhao, Hongyuan Zha
  \item Introduced me to the keyword ``Collaborative Filtering''
  \item The basic idea of this paper is a method of matrix factorization where some factors are known and others are not.
  \item Factors with regularization to the known filter elements.
  \item Could be a useful way for me to look for common factors if extended to tesnors.
\end{itemize}

\subsection{integrating content based filtering with collaborative filtering}
\begin{itemize}
  \item {\bf Title:} Integrating content-based filtering with collaborative filtering using co-clustering with augmented matrices
  \item {\bf Author:} Meng-Lun Wu, Chia-Hui Chang, Rui-Zhe Liu
  \item Novel method, CCAM, which combines collaborative filtering with content based systems.
  \item Clusters based on content similarity and user input.
  \item May be a useful inspiration for an augmented decomposition model, though not directly applicable to me.
\end{itemize}

\subsection{Low n rank tensor recovery based on multi-linear augmented lagrange}

This paper presents an approach to recover a Low order component from a tensor.  The basic idea is to solve the low rank recovery method via a novel 
application of Legrange Multipliers.  This reduces the noise in the 
recovered tensor by separating the tensors into their low rank and sparse components, satisfying:

\[
\mathrm{min}_{L,S} : ||L||_* + \lambda ||S||_1 
\]

Where $A=L+S$.  Here $A$ is the matrix to be recovered, $L$ is the low-rank 
component we wish to recover, and $S$ is the sparse component of A.  $\lambda$ is a positive weighted parameter.

The technique describe effectively separates the low order component from
the sparse and is demonstrated to work in several image noise removal
applications.

This paper may be interesting for general usage of Tensors, and perhaps
will inspire decomposition work, but it does not appear to be very relevent
to my work at this time.

\subsection{hashmann-wpppfac0.pdf}
This is the {\em the original} PARAFAC Paper.  In general, PARAFAC will be
provided to me by a python library, however this may be a useful read 
when modifiying the algorithm.

Everyone cites this, so should I.

\subsection{splatt.pdf}
This is a description of a parallel parafac implimentation.  It describes
an algorithm implemented in a package of the same name.   SPLATT looks 
promising, but in my attempts to use it, it always results in NaN.

Speed is not necessarily in my present objectives, so I probably won't
revisit SPLATT.

\subsection{wikisent.pdf}
Simple sentiment analysis scheme.  Not really relevant to me.

\subsection{parcube-sparse tensor.pdf}
Technique for decomposition of sparse tensors.  Could be handy for
making a speedy version of my algorithm!

\subsection{Fully sparse topic models.pdf}
A novel method of topic modelling for an entire corpus of documents at a
time.  They present a method of estimating $\theta$, which is the 
infulence of each topic on the document.  This could be very 
useful to me, and closer reading/notes are definitely in order!

\subsection{1-s2.0-s0169743997000324-parafac.pdf}
Just af tutorial over various uses of Parafac.  Good information, though
not particularly citable.

\subsection{icdm2005-Liu-Text-Vectors-To-Tensors.pdf}

\subsection{30204713-Mission Survival Corpus.pdf}

\subsection{p775-sun-models Influence Analysis.pdf}
short intro piece for a tutorial

good citations though

\subsection{p559-li-Influence Heterogeneous.pdf}

probability network preselected features.
very short and little more than prob net.

\subsection{p171-liu-Influence Maximiztion Linear Approach.pdf}
a nice and very technical article about 
the social influence maximization problem
with an efficient solution.

there is nothing presented about feature identifocation

There is a good description of a fast linear
approach for discovering seed nodes in a network
though.  Elements of this approach could work
for finding seminal works in my networks.

\section{p207-argawal}

\section{a7-aziz}

\subsection{p271-dong-influence Model Functional Roles.pdf}

\subsection{p33 kasthuriathna-Influence Modeling Boundary.pdf}

\subsection{p46-herzig-Author-Reader Influence.pdf}

\subsection{p15-jiang-dynamic Topic-Citation Influence.pdf}

\subsection{surveyTensors-Idl-09-34.pdf}

\subsection{tensordecompositions.pdf}

\subsection{Tensors TM2002211716.pdf}

\subsection{p113-blei-dynamic-topic.pdf}

\subsection{1-s2.0-S0925231212009083-main-tensor.pdf}

\subsection{1-s2.0-S0378437102007367-main.pdf}

\subsection{0910.3529-Citation Statistics.pdf}

\subsection{1145-Discovering-Relavent-Scientific-Literature.pdf}

\subsection{p50-buneman.pdf}
\end{document}
