\documentclass{article}
\title{Proposal: Influence Modeling Through Tensor Decomposition}
\author{Robert Lowe}

\usepackage{epigraph}
\usepackage{cite}

%some custom tensor commands
\newcommand{\tens}[1]{\mathcal{#1}}
\newcommand{\ntens}[1]{\hat{\tens{#1}}}

\begin{document}
\maketitle

\section{Introduction}
\begin{quote}
\textit{Nam cum pictor praecogitat quae facturus est, habet quidem in intellectu sed nondum intelligit esse quod nondum fecit.}

-- Anselm of Canterbury~\cite{anselm}
\end{quote}

The written word can be thought of as one of the chief accomplishments of mankind.  By recording human thought, our species has managed to make steady progress toward some as yet unknown end, always being able to make advances by utilizing the works of those that came before us.  In fact, written documents can even lend a certain degree of immortality to their authors in that they remain long after the author dies.  As is the case with all human thought, written words are not entirely the work of their author.

Being an extension of the human mind, written documents share in the epistymological mystery of thought.  To paraphrase Anselm of Canterbury, that which we create must first be in the intellect.  So then, that begs the question of where these intellectual insights come from.  Of course, this has been the topic of much debate by many philosophers of the past, and the present work will make no effort to solve this.  Instead, we will make an appeal to Anselm's argument that everything that is in the intellect is either conceived within the intellect or it stems from some source which exists in outside reality.  The present work will also diverge wildly from what Anselm originally set out to accomplish.  The formulation of conception from internal and external sources stated above was originally the first step in the earliest ontological argument which proved the existence of God.  What the present work will attempt to do is much more difficult, involving something much more elusive than the creator of the universe.  What we will be searching for is a reliable way to model influence and contribution of human thought.


\section{Bibliography}
\bibliography{sources}{}
\bibliographystyle{plain}

\end{document}