\documentclass{article}
\title{Proposal: Influence Modeling Through Tensor Decomposition}
\author{Robert Lowe}

\usepackage{epigraph}
\usepackage{cite}

%some custom tensor commands
\newcommand{\tens}[1]{\mathcal{#1}}
\newcommand{\ntens}[1]{\hat{\tens{#1}}}

\begin{document}
\maketitle

\section{Introduction}
\begin{quote}
\textit{Nam cum pictor praecogitat quae facturus est, habet quidem in intellectu sed nondum intelligit esse quod nondum fecit.}

-- Anselm of Canterbury~\cite{anselm}
\end{quote}

The written word can be thought of as one of the chief accomplishments
of mankind.  By recording human thought, our species has managed to
make steady progress toward some as yet unknown end, always being able
to make advances by utilizing the works of those that came before us.
In fact, written documents can even lend a certain degree of
immortality to their authors in that they remain long after the author
dies.  As is the case with all human thought, written words are not
entirely the work of their author.

Being an extension of the human mind, written documents share in the
epistymological mystery of thought.  To paraphrase Anselm of
Canterbury, that which we create must first be in the intellect.  So
then, that begs the question of where these intellectual insights come
from.  Of course, this has been the topic of much debate by many
philosophers of the past, and the present work will make no effort to
solve this.  Instead, we will make an appeal to Anselm's argument that
everything that is in the intellect is either conceived within the
intellect or it stems from some source which exists in outside
reality.  The present work will also diverge wildly from what Anselm
originally set out to accomplish.  The formulation of conception from
internal and external sources stated above was originally the first
step in the earliest ontological argument which proved the existence
of God.  What the present work will attempt to do is much more
difficult, involving something much more elusive than the creator of
the universe.  What we will be searching for is a reliable way to
model influence and contribution of human thought.

\subsection{Influence Modelling}
The current proposed work seeks to create a novel method of influence
modelling through tensor decomposition. Influence modelling in this
context is a model which accomplishes the following goals:
\begin{enumerate}
\item The model determines a weight for the influence one document
  exerts on another.
\item The model determines a weight for the contribution of an author
  to their own document. 
\item The model identifies what concepts are influencing factors and
  what concepts are original to the document.
\item Provide a weight for a document's influence over the entire
  corpus to which it belongs.
\end{enumerate}
This sort of model can be used for a variety of tasks, such as
document recommendation, plagerism detection, and author impact
evaluation.  

Stated more formally, given a document $D$, the proposed model creates
the parameters:
\[
\langle I, C, F \rangle 
\]
Each of these paramters are a pair of arrays $(W, T)$ where $W$ is an
array of weights and $T$ is an array of topics such that $w_i$ is the
weight of topic $t_i$.

\subsection{Approach Overview}
The proposed approach to influence modeling is based on the premise
that concepts within a corpus of documents are either created by the
authors of documents or are transferred from one document to another.
The view is that a document within a corpus can be modeled as the sum
of these parts.  Suppose that a document $D_1$ is composed of a set of
concepts $C_1$, and that document $D_2$ is composed of a set of
concepts $C_2$.  The intersection $C_1 \cap C_2$ would therefore
represent the common concepts between these two documents.  If we
suppose that $D_2$ was influenced by $D_1$, then it is a reasonable
assumption that the set of influencing concepts is given as $I \subseteq
C_1 \cap C_2$.  

Having arrived at the set of influencing concepts, $I_1$, a natural
question would be to determine the strength of the relationship.
Speaking again in broad generalities, suppose we have some set $W_2$
which weights the concepts $c \in C_2$ according to importance to the
document.  (This could, of course, be computed in many different
ways!) For convenience, let $\sum W_2 = 1$.  Now, because $I_1 \subseteq
C_1 \cap C_2$, it follows that $I_1 \subseteq C_2$.  Therefore, every
concept $c\in I$ must have a corresponding $w \in W_2$.  Moreoever, we
can build the set $W_i \subseteq W_2$ by extracting these
corresponding elements, and we can assign a weight to the influence of
$D_1$ by simple summation.

\section{Bibliography}
\bibliography{sources}{}
\bibliographystyle{plain}

\end{document}
