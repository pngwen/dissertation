\section{Proposed Approach}
The basic approach to the proposed problem is to use a combination of
Particle Filter based prognostics and a physical model of the system
under observation.  This is an extension of Marcos Orchard's approach
to PF based prognostics.  (See the related work section for details of
Orchard's existing body of work.)

In addition to the prognostic step, the present proposal include
a method by which prognostics can be applied in the control loop.
There are two main goals present here.  First, there is low level
control considerations where control parameters are modified in
response to predicted future state of the system.  The secondary is
a high level planning considerations wherein mission parameters are
modified in response to predicted future loss of system capabilities.

In the case of low level control modification, the objective is to
extend the remaining useful life of the system and preserve all of its
capabilities for as long as possible.  In fact there may be situations
in which control reconfiguration can completely avert an impending
predicted fault.  This objective is, of course, not always achievable,
and some system faults may not be avoidable.  In these circumstances,
low level control can still be modified in such a way to extend the
time before a loss of capabilities occurs.

Given that some loss of capabilities is unavoidable, there still
exists a need to replan tasks so as to maximize mission utility.  This
is where the high level replanning system comes into play.  In this
part of the system, prognostic information about impending failures is
used to redefine mission constraints.  

The key cotributions the proposed work will provide are:
\begin{itemize}
\item A novel control scheme using prognostic data in the control
loop.
\item A novel prognostic model utilizing both particle filters and
physics simulation.
\item A novel approach to future capabilities estimation.
\item A novel mission planning scheme using future capabilities
estimation.
\end{itemize}
